% --------------------------------------------------------------
% This is all preamble stuff that you don't have to worry about.
% Head down to where it says "Start here"
% --------------------------------------------------------------
 
\documentclass[12pt]{article}
 
\usepackage[margin=1in]{geometry} 
\usepackage{amsmath,amsthm,amssymb}
\usepackage{graphicx}
\usepackage{mathtools}
\usepackage{tensor}

\newcommand{\levicivita}{}% initialize
\def\levicivita#1#{\tensor#1{\epsilon}}

\newcommand{\N}{\mathbb{N}}
\newcommand{\Z}{\mathbb{Z}}
\newcommand{\R}{\mathbb{R}}
\newcommand{\I}{\mathbb{I}}

\newenvironment{theorem}[2][Theorem]{\begin{trivlist}
\item[\hskip \labelsep {\bfseries #1}\hskip \labelsep {\bfseries #2.}]}{\end{trivlist}}
\newenvironment{lemma}[2][Lemma]{\begin{trivlist}
\item[\hskip \labelsep {\bfseries #1}\hskip \labelsep {\bfseries #2.}]}{\end{trivlist}}
\newenvironment{exercise}[2][Exercise]{\begin{trivlist}
\item[\hskip \labelsep {\bfseries #1}\hskip \labelsep {\bfseries #2.}]}{\end{trivlist}}
\newenvironment{problem}[2][Problem]{\begin{trivlist}
\item[\hskip \labelsep {\bfseries #1}\hskip \labelsep {\bfseries #2.}]}{\end{trivlist}}
\newenvironment{question}[2][Question]{\begin{trivlist}
\item[\hskip \labelsep {\bfseries #1}\hskip \labelsep {\bfseries #2.}]}{\end{trivlist}}
\newenvironment{corollary}[2][Corollary]{\begin{trivlist}
\item[\hskip \labelsep {\bfseries #1}\hskip \labelsep {\bfseries #2.}]}{\end{trivlist}}

\newcommand\norm[1]{\left\lVert#1\right\rVert}

\newenvironment{solution}{\begin{proof}[Solution]}{\end{proof}}
 
\begin{document}
 
% --------------------------------------------------------------
%                         Start here
% --------------------------------------------------------------
 
\title{Chapter 2 Problems}
\author{Panya Sukphranee\\ %replace with your name
A Mathematical Introduction to Robotic Manipulation}
\date{}
\maketitle

    \begin{enumerate}
        \item Let $ a,b,c, \in \R^3 $ with the standard basis $\{ \vec{e_1}, \vec{e_2}, \vec{e_3} \}$. Using Einstein summation notation, \\
            \begin{enumerate}
                \item[(a)] \\
                    \begin{align}
                        \vec{a} \cdot (\vec{b} \times \vec{c}) &= (a^i \vec{e_i}) \cdot (\levicivita{^{l}_{jk}}b^jc^k) \vec{e_l} \\
                                                    &= \delta_{il} a^i \levicivita{^{l}_{jk}}b^jc^k \\
                                                    &= \sum_i \levicivita{^{i}_{jk}}a^i b^j c^k \\
                                                    &= \levicivita{_{ijk}}a^i b^j c^k \\
                                                    &= \levicivita{_{kij}}a^i b^j c^k \\
                                                    &= (\vec{a} \times \vec{b})_k c^k \\
                                                    &= (\vec{a} \times \vec{b}) \cdot \vec{c} 
                    \end{align}
                \item[(b)]
                    \begin{align}
                        \vec{a} \times (\vec{b} \times \vec{c}) &= \levicivita{^{i}_{jk}}a^j(\levicivita{^{k}_{lm}}b^l c^m) \vec{e_i} \\
                                                    &= (\levicivita{^{i}_{jk}}\levicivita{^{k}_{lm}} a^j b^l c^m) \vec{e_i} \\
                                                    &= \sum_i (\delta_{il}\delta_{jm} - \delta_{im}\delta_{jl})(a^j b^l c^m) \vec{e_i}\\
                                                    &= \sum_j (a^j b^i c^j)\vec{e_i} - (a^j b^j c^i)\vec{e_i}\\
                                                    &= (\vec{a} \cdot \vec{c})\vec{b} - (\vec{a} \cdot \vec{b})\vec{c}
                    \end{align}
            \end{enumerate}
            
        \item Let $g,h \in SE(3)$, $\bar{g} = \begin{bmatrix} 
                                                R_g & p_g \\ 
                                                0 & 1
                                                \end{bmatrix}$ and 
                                      $\bar{h} = \begin{bmatrix} 
                                                R_h & p_h \\ 
                                                0 & 1
                                                \end{bmatrix}$ 
                \begin{enumerate}
                    \item Closure
                        \begin{align}
                            \bar{g} \bar{h} &= \begin{bmatrix} 
                                                R_g & p_g \\ 
                                                0 & 1
                                            \end{bmatrix} 
                                            \begin{bmatrix} 
                                                R_h & p_h \\ 
                                                0 & 1
                                            \end{bmatrix} \\
                                     &= \begin{bmatrix}
                                            R_g R_h & R_g p_h + p_g\\ 
                                            0 & 1
                                        \end{bmatrix}
                        \end{align}
                        Therefore, $gh = (R_g R_h, R_g p_h + p_g) \in SE(3)$
                    \item Identity element exists 
                    
                        Let $e = (\I_{3x3}, 0) \in SE(3)$,
                        \begin{align}
                            \bar{g} \bar{e} &= \begin{bmatrix} 
                                            R_g & p_g \\ 
                                            0 & 1
                                        \end{bmatrix} 
                                        \begin{bmatrix} 
                                            \I_{3x3} & 0 \\ 
                                            0 & 1
                                        \end{bmatrix} \\
                                     &= \begin{bmatrix} 
                                            R_g & p_g \\ 
                                            0 & 1
                                        \end{bmatrix} \\
                                    &= \bar{g}
                        \end{align}
                    
                                    and
                        \begin{align}
                             \bar{e} \bar{g} &= \begin{bmatrix} 
                                            \I_{3x3} & 0 \\ 
                                            0 & 1
                                        \end{bmatrix}
                                        \begin{bmatrix} 
                                            R_g & p_g \\ 
                                            0 & 1
                                        \end{bmatrix} \\
                                     &= \begin{bmatrix} 
                                            R_g & p_g \\ 
                                            0 & 1
                                        \end{bmatrix} \\
                                    &= \bar{g} 
                        \end{align}
                        
                        Thus, $ge = eg = g \Rightarrow e \in SE(3)$ is the identity element.
                        
                    \item Inverse Exists
                        Let $g \in SE(3)$, $\bar{g} = \begin{bmatrix} 
                                                R_g & p_g \\ 
                                                0 & 1
                                                \end{bmatrix}$. \\
                        Consider $\bar{h} = \begin{bmatrix} 
                                                R_g^\top & -R_g^\top p_g \\ 
                                                0 & 1
                                                \end{bmatrix} \in \R_{4x4}$\\
                                    
                        \begin{align}
                            \bar{g} \bar{h} &=
                            \begin{bmatrix} 
                                R_g & p_g \\ 
                                0 & 1
                            \end{bmatrix} 
                            \begin{bmatrix} 
                                R_g^\top & -R_g^\top p_g \\ 
                                0 & 1
                            \end{bmatrix} \\
                            &= \begin{bmatrix} 
                                R_g R_g^\top & - R_g R_g^\top p_g + p_g \\ 
                                0 & 1
                            \end{bmatrix} \\
                            &= \begin{bmatrix} 
                                \I_{3x3} & 0 \\ 
                                0 & 1
                            \end{bmatrix} \\
                            &= \bar{g}
                        \end{align}
                        On the other hand,
                        \begin{align}
                            \bar{h} \bar{g}  &=
                            \begin{bmatrix} 
                                R_g^\top & -R_g^\top p_g \\ 
                                0 & 1
                            \end{bmatrix} 
                            \begin{bmatrix} 
                                R_g & p_g \\ 
                                0 & 1
                            \end{bmatrix}  \\
                            &= \begin{bmatrix} 
                                R_g^\top R_g  & R_g^\top p_g - R_g^\top p_g \\ 
                                0 & 1
                            \end{bmatrix} \\
                            &= \begin{bmatrix} 
                                \I_{3x3} & 0 \\ 
                                0 & 1
                            \end{bmatrix} \\
                            &= \bar{g}
                        \end{align}
                        
                        Since $\bar{h} = \bar{g}^{-1}$, $g^{-1} = (R_g^\top,-R_g^\top p_g  )$.
                        
                    \item Associativity 
                    
                        Let $f,g,h \in SE(3)$, then
                        $\bar{f} = \begin{bmatrix} 
                                        R_f & p_f \\ 
                                        0 & 1
                                    \end{bmatrix}$ 
                                    $\bar{g} = \begin{bmatrix} 
                                        R_g & p_g \\ 
                                        0 & 1
                                    \end{bmatrix}$ 
                                    $\bar{h} = \begin{bmatrix} 
                                        R_h & p_h \\ 
                                        0 & 1
                                    \end{bmatrix}$ 
                        
                        \begin{align} 
                            (\bar{f}\bar{g})\bar{h} &= \big(\begin{bmatrix} 
                                                            R_f & p_f \\ 
                                                            0 & 1
                                                        \end{bmatrix}
                                                        \begin{bmatrix} 
                                                            R_g & p_g \\ 
                                                            0 & 1
                                                        \end{bmatrix}\big)
                                                        \begin{bmatrix} 
                                                            R_h & p_h \\ 
                                                            0 & 1
                                                        \end{bmatrix}\\
                                                    &= \big(\begin{bmatrix}
                                                                R_f R_g & R_f p_g + p_f\\ 
                                                                0 & 1
                                                            \end{bmatrix}\big)
                                                        \begin{bmatrix} 
                                                            R_h & p_h \\ 
                                                            0 & 1
                                                        \end{bmatrix}\\
                                                    &= \begin{bmatrix}
                                                                R_f R_g R_h & R_f R_g p_h + R_f p_g + p_f\\ 
                                                                0 & 1
                                                        \end{bmatrix}
                        \end{align}
                        
                        \begin{align} 
                            \bar{f}(\bar{g}\bar{h}) &= \begin{bmatrix} 
                                                            R_f & p_f \\ 
                                                            0 & 1
                                                        \end{bmatrix}
                                                        \big(\begin{bmatrix} 
                                                            R_g & p_g \\ 
                                                            0 & 1
                                                        \end{bmatrix}
                                                        \begin{bmatrix} 
                                                            R_h & p_h \\ 
                                                            0 & 1
                                                        \end{bmatrix}\big)\\
                                                    &= \begin{bmatrix} 
                                                            R_f & p_f \\ 
                                                            0 & 1
                                                        \end{bmatrix}
                                                        \big(\begin{bmatrix}
                                                                R_g R_h & R_g p_h + p_g\\ 
                                                                0 & 1
                                                            \end{bmatrix}\big)\\
                                                    &= \begin{bmatrix}
                                                                R_f R_g R_h & R_f R_g p_h + R_f p_g + p_f\\ 
                                                                0 & 1
                                                        \end{bmatrix}
                        \end{align}
                        
                        Thus, $(fg)h = f(gh)$.
                \end{enumerate}
        \item Properties of Rotation Matrices
            \begin{enumerate}
                \item Find the eigenvalues and eigenvectors of $\widehat{w} = \begin{bmatrix}
                                                                            0 & -\omega^3 & \omega^2 \\ 
                                                                            \omega^3 & 0 & -\omega^1\\ 
                                                                            -\omega^2 & \omega^1  & 0
                                                                            \end{bmatrix}$, $\norm{\vec{w}} = 1.$
                    \begin{align}
                        |\widehat{w} - \lambda \I|  &=  \begin{vmatrix}
                                                            - \lambda & -\omega^3 & \omega^2 \\ 
                                                            \omega^3 & - \lambda & -\omega^1\\ 
                                                            -\omega^2 & \omega^1  & - \lambda
                                                        \end{vmatrix} \\
                                                    &= - \lambda(\lambda^2 + \omega_1^2) 
                                                        - \omega_3(\lambda \omega_3 - \omega_1 \omega_2)
                                                        - \omega_2(\omega_1 \omega_3 + \lambda \omega_2)
                                                    \intertext{(being in Euclidean $\R^3$, we can lower the $\omega$ indices for simplicity)} \\
                                                    &= - \lambda^3 - \lambda\omega_1^2
                                                        - \lambda \omega_3^2 + \omega_1 \omega_2 \omega_3
                                                        - \omega_1 \omega_2 \omega_3 - \lambda \omega_2^2 \\
                                                    &= - \lambda^3 - \lambda \norm{\omega}^2 \\
                                                    &= - \lambda^3 - \lambda \\
                                                    &= 0
                    \end{align}
                    implies $\lambda=0, \pm i$. We find the corresponding eigenvectors.\\
                    \begin{enumerate}
                        \item $\lambda = 0$. Solve $\widehat{w} \vec{x} = \vec{0}$. Wlog, $\omega_3 \neq 0$, row reduce the following:
                            \begin{align*}
                                \begin{bmatrix}
                                    0 & -\omega_3  & \omega_2\\ 
                                    \omega_3 & 0  & -\omega_1\\ 
                                    -\omega_2 & \omega_1 & 0
                                \end{bmatrix} &\rightarrow
                                \begin{bmatrix}
                                    0 & 1  & -\frac{\omega_2}{\omega_3}\\ 
                                    1 & 0  & -\frac{\omega_1}{\omega_3}\\ 
                                    -\omega_2 & \omega_1 & 0
                                \end{bmatrix} \rightarrow\\
                                \begin{bmatrix}
                                    1 & 0  & -\frac{\omega_1}{\omega_3}\\ 
                                    0 & 1  & -\frac{\omega_2}{\omega_3}\\
                                    -\omega_2 & \omega_1 & 0
                                \end{bmatrix} &\rightarrow
                                \begin{bmatrix}
                                    1 & 0  & -\frac{\omega_1}{\omega_3}\\ 
                                    0 & 1  & -\frac{\omega_2}{\omega_3}\\
                                    0 & \omega_1 & -\frac{\omega_1 \omega_2}{\omega_3}
                                \end{bmatrix} \rightarrow\\
                                \begin{bmatrix}
                                    1 & 0  & -\frac{\omega_1}{\omega_3}\\ 
                                    0 & 1  & -\frac{\omega_2}{\omega_3}\\
                                    0 & 0 & 0
                                \end{bmatrix} &\rightarrow t \begin{bmatrix}
                                                                1\\ 
                                                                1\\ 
                                                                1
                                                            \end{bmatrix}
                            \end{align*}
                    \end{enumerate}
            \end{enumerate}
        \item Properties of skew-symmetric matrices
            \begin{enumerate}
                \item It was easier to solve part (b) first and use those results for this problem.
                    Let $\vec{x} \in \R^3$. 
                    \begin{align}
                        (R \widehat{w} R^\top) \vec{x} &= R \widehat{w} (R^\top \vec{x}) \\
                                                &= R (\vec{w} \times (R^\top \vec{x})) \\
                                                &= R \vec{w} \times R R^\top \vec{x} \\
                                                &= (\widehat{R \vec{w}}) \vec{x}
                    \end{align}
                    
                \item If $R \in SO(3)$ and $v,w \in \R^3$, then $R(\vec{v} \times \vec{w} )=( R \vec{v}) \times (R \vec{w})$. We'll show equality by comparing the $l-th$ component of each side. The result utilizes the relationship between the dot product and multiplication by transpose.
                \begin{align}
                    [R(\vec{v} \times \vec{w} )]^l &= \vec{e_l} \cdot R(\vec{v} \times \vec{w} ) \\
                                            &= \vec{e_l} ^\top R(\vec{v} \times \vec{w} ) \\
                                            &= (R^\top \vec{e_l})^{\top} (\vec{v} \times \vec{w} ) \\
                                            &= (R^\top \vec{e_l}) \cdot (\vec{v} \times \vec{w}) \label{comment_1}\\
                                            \intertext{This is the determinant of matrix with columns $R^\top \vec{e_l}, \vec{v},$ and $\vec{w}$, respectively. } 
                                            &= det([R^\top \vec{e_l} , \vec{v} ,\vec{w}]) \\
                                            \intertext{Since $R^\top R = \I$,} 
                                            &= det(R^\top R [R^\top \vec{e_l} , \vec{v} , \vec{w}]) \\
                                            &= det(R^\top [\vec{e_l} , R \vec{v} , R \vec{w}]) \\
                                            &= det(R^\top) det([\vec{e_l} , R \vec{v} , R \vec{w}])\\
                                            &= det([\vec{e_l} , R \vec{v} , R \vec{w}])\\
                                            &= \vec{e_l} \cdot (R \vec{v} \times R \vec{w}) \\
                                            &= (R \vec{v} \times R \vec{w})^l
                \end{align}
                Thus, $R(\vec{v} \times \vec{w} )=( R \vec{v}) \times (R \vec{w})$
            \end{enumerate}
        \item Cayley Parameters
        
            Cayley Parameterization, like the exponential map, is a mapping from $so(3)$ to $SO(3)$. In this problem we show that $R_a = (\I - \widehat{a})^{-1}(\I + \widehat{a})$ is indeed an element of $SO(3)$, given $\widehat{a} \in so(3)$. The derivation of this mapping can be found going in the opposite direction. i.e. by letting $R_a \in SO(3)$ and showing $\widehat{a} \in so(3)$; this involves using the diagonals of the parallelogram formed by some vector $\vec{v}$ and its transformation $R_a{\vec{v}}$.
            
            \begin{enumerate}
                \item Show $R_a = (\I - \widehat{a})^{-1}(\I + \widehat{a}) \in SO(3)$. 
                
                    Since, $\widehat{a}$ is anti-symmetric, $\widehat{a} ^\top = - \widehat{a} $. Therefore, $(\I \pm \widehat{a})^\top = (\I \mp \widehat{a})$. Recall from Linear Algebra that \[(AB)^{-1}=B^{-1}A^{-1}\] \[ (AB)^{\top}=B^{\top}A^{\top} \]  \[ (A^{-1})^{\top} = (A^{\top})^{-1}\]
                    
                    Note that the transformations $(\I - \widehat{a})^{-1}$ and $(\I + \widehat{a})$ commute. To show this, we write $(\I + \widehat{a})$ in terms of $(\I - \widehat{a})$.
                    \begin{align*}
                        (\I - \widehat{a})^{-1}(\I + \widehat{a}) &= - (\I - \widehat{a})^{-1} (\I + \widehat{a}) \\
                                                        &= - (\I - \widehat{a})^{-1} (-2\I + \I - \widehat{a})\\
                                                        &= -  (\I - \widehat{a})^{-1} (-2\I + (\I - \widehat{a}))\\
                                                        &=  (\I - \widehat{a})^{-1} (2\I - (\I - \widehat{a}))\\
                                                        &= 2(\I - \widehat{a})^{-1} - (\I - \widehat{a})^{-1} (\I - \widehat{a})\\
                                                        &= 2(\I - \widehat{a})^{-1} - \I \\
                                                        &= 2(\I - \widehat{a})^{-1} - (\I - \widehat{a}) (\I - \widehat{a})^{-1}\\
                                                        &= [2 \I - (\I - \widehat{a})] (\I - \widehat{a})^{-1} \\
                                                        &= (\I + \widehat{a}) (\I - \widehat{a})^{-1}
                    \end{align*}
                    Now,     
                    \begin{align}
                        R_a^{\top} R_a &= [(\I - \widehat{a})^{-1}(\I + \widehat{a})]^{\top}(\I - \widehat{a})^{-1}(\I + \widehat{a}) \\
                                        &= (\I + \widehat{a})^{\top}[(\I - \widehat{a})^{-1}]^{\top}(\I - \widehat{a})^{-1}(\I + \widehat{a}) \\
                                        &= (\I - \widehat{a}) (\I + \widehat{a})^{-1} (\I - \widehat{a})^{-1}(\I + \widehat{a})\\
                                        &= (\I - \widehat{a}) (\I + \widehat{a})^{-1} (\I + \widehat{a})(\I - \widehat{a})^{-1}\\
                                        &= \I
                    \end{align}
                    Similarly, $R_a R_a^{\top}=\I$. \textbf{Thus, $R_a$ is orthogonal}. Now, we will show \[ det R_a = 1 \]
                    \begin{align}
                        |(\I - \widehat{a})^{-1}(\I + \widehat{a})| &= |(\I - \widehat{a})^{-1}||(\I + \widehat{a})|\\
                                                            &= |(\I - \widehat{a})^{-1}||(\I + \widehat{a})^{\top}|\\
                                                            &= |(\I - \widehat{a})^{-1}||(\I - \widehat{a})|\\
                                                            &= 1
                    \end{align}
                    Since, $|R_a| = 1$, \textbf{$R_a \in SO(3). \qed$}
            \end{enumerate}
    \end{enumerate}

\end{document}
